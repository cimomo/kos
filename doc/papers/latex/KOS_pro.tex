\documentclass[dvips,11pt]{article}

\usepackage{graphicx}      % extended graphics package
\usepackage{epsfig}        % wrapper for graphicx package
%\usepackage{bbm}
\usepackage{amsmath,amsthm,amsfonts,latexsym}
\newcommand{\etal}{{\it et al}.$\:$}
\newcommand{\eg}{{\it e}.{\it g}.$\:$}
\newcommand{\cf}{{\it cf}.$\:$}
\newcommand{\ie}{{\it i}.{\it e}.$\:$}
\setlength{\oddsidemargin}{0in}
\setlength{\textwidth}{5.8in}
\setlength{\topmargin}{0in}
\setlength{\textheight}{8in}
\setlength{\parskip}{.10in}


\newtheorem{theorem}{Theorem}[section]
\newtheorem{lemma}[theorem]{Lemma}
\newtheorem{proposition}[theorem]{Proposition}
\newtheorem{corollary}[theorem]{Corollary}
\newtheorem{definition}[theorem]{Definition}

%\newenvironment{proof}[1][Proof]{\begin{trivlist}
%\item[\hskip \labelsep {\bfseries #1}]}{\end{trivlist}}

%\newenvironment{definition}[1][Definition]{\begin{trivlist}
%\item[\hskip \labelsep {\bfseries #1}]}{\end{trivlist}}
\newenvironment{example}[1][Example]{\begin{trivlist}
\item[\hskip \labelsep {\bfseries #1}]}{\end{trivlist}}
\newenvironment{remark}[1][Remark]{\begin{trivlist}
\item[\hskip \labelsep {\bfseries #1}]}{\end{trivlist}}

%\newcommand{\qed}{\nobreak \ifvmode \relax \else
%\ifdim\lastskip<1.5em \hskip-\lastskip
%\hskip1.5em plus0em minus0.5em \fi \nobreak
%\vrule height0.75em width0.5em depth0.25em\fi}


\newcounter{counter}

\begin{document}



\title{\large{\bf{K-OS PROJECT PROPOSAL}}
\\
\normalsize{\bf{CS134c, Spring 2003}}}

\author{\it{\small{Kai Chen}}}

\date{}


\maketitle


\section{\textbf{\sc{Overview}}}
We propose to write an operating system {\bf K-OS} ({\bf K}ai's {\bf
O}perating {\bf S}ystem) from scratch. {\bf K-OS} is designed for the
IA-32 Intel architecture, and will be developed and tested on
machines with Intel Pentium processors. In this proposal, we carefully
specify the goals we hope to achieve and describe the approach we
will take. We also give a rough timeline for the completion of each of
the several phases.


\section{\textbf{\sc{Objective}}}
From this project, we hope to gain experience with low-level
system design, interfacing with hardware and to understand concepts
and details of various aspects of the operating system. 

For this class, we aim for a rather complete implementation of the basic
features of an operating system in a {\it simple} manner. We hope to
design {\bf K-OS} in such a way that it can later be extended in
various directions when researchers need a simple enough system that
they fully understand to support researches in different fields.

Features we plan to support for {\bf K-OS} and the related components 
include:

\begin{itemize}

\item {\bf Boot Sector}: {\bf K-OS} will of course need to be able to
  boot itself at the very least.
\item {\bf Interrupt Handler}: Necessary for supporting I/O, system
  calls, and handling exceptions.
\item {\bf Memory management}: We need to at least keep track of
  which memory is being used and which is free. We also need to
  provide simple mechanisms to allocate and free memory. We also hope
  to support some simplistic virtual memory with paging if time
  permits. 
\item {\bf I/O}: We need low level mechanisms to generate and handle
  interrupts and to read from and write to hardware ports. We will
  also provide highlevel I/O routines for system and application
  usage. 
\item {\bf Process Management}: Some notion of ``process'' is
  necessary. We also hope to support some simple multi-tasking. 
\item {\bf Application}: We want to be able to run applications. The
  kernel needs to load application into memory. The application will
  interface with the kernel via system calls. We might also want to
  implement some standard library functions for {\bf K-OS}.
\item {\bf File System} (Optional): We hope to implement something 
  like DOS FAT12 file system for {\bf K-OS}.

\end{itemize}

In the next section, we describe our approach to implement each
component, and give a rough timeline for each phase.



\section{\textbf{\sc{Approach}}}

We have a total of 10 weeks for this project, including design,
implementation and testing. The first 2 weeks were
spent on researching and designing. As stated earlier, our goal here
is to keep everything simple. Next we describe detailedly how we would
approach each of the proposed components.

\subsection{\textbf{Boot Sector}}
The {\bf K-OS} kernel will be loaded from a floppy. We will need to
write the boot loader to the first sector (boot sector) of a
floppy. After performing {\it POST}, BIOS will copy the boot loader
into memory location $0x7C00$. From there the control is transfered to
the boot loader. In the boot sector we will enable the $A20$ line to
allow {\bf K-OS} to access the entire memory under protected mode. It
will also set up the Global Descriptor Table (GDP), and load rest of
the kernel. Before switching to protected mode, we might want to make
sure the processor is $80386$ or above. We also need to keep in mind
that the boot loader must be $512$ bytes in size and end with
$0xAA55$. We will want to display something on the screen to show
progress. 

We plan to finish this part by $4/25$.


\subsection{\textbf{Interrupt Handler}}
We will need to set up the Interrupt Descriptor Table (IDT) with
interrupt service routines. We at least need to handle keyboard
interrupt, video interrupt, clock interrupt, page fault interrupt if
we use paging, and interrupts for system
calls. A central dispatching function will map different interrupts to
the corresponding handlers. Each handler will be implemented in some
high level language (likely, C) with an assembly wrapper. The assembly
wrapper is necessary when we need to manipulate registers differently
from C or when we need to call the IRET instruction. We will also need
to initialize the Programmable Interrupt Controller (PIC) to start
receiving interrupts. 

We plan to set up this part by $5/2$. We may add more handlers as the
project evolves.


\subsection{\textbf{Memory Management}}
We need some basic memory management before we can move on. Two
different schemes are necessary, one for kernel space, and one for
user space. (Best illustrated with $kmalloc()$/$kfree()$ and
$malloc()$/$free()$.) We hope to support limited virtual memory with
paging. We need some efficient data structure to keep track of raw
memory usage. The kernel space allocator will allocate memory and
return physical address. The user space allocator will allocate memory
from the application's memory space, which is protected by paging. If
there is no free memory, it will call the kernel allocator and
populate its page table with new memory. 

This part can get overly complicated. We will make sure to support
very basic memory management first before worrying about virtual
memory. We plan to finish this part by $5/11$


\subsection{\textbf{I/O}}
At the low level, we work with I/O address space. Unlike the main
physical memory, here we use instructions IN and OUT to read from and
write to I/O ports. 

Printing on the screen is fairly straightforward. We need to keep track of
the position of the cursor. The video screen is mapped to physical
memory $0xb8000$. The text mode is $80$ columns by $25$ rows. Each
character is represented by $2$ bytes. One for ASCII value and one for
attribute. After outputing a character, we need to advance the
cursor. The screen shall scroll if the cursor passes the bottom of the
screen. 

Keyboard is interrupt driven. We keep a buffer which the interrupt
handler can write to. When we have processes, we also need to take
care of those processes that are waiting on keyboard events.

At a higher level, we might want to implement some functions similar
to $printf()$ to handle I/O, since we cannot use standard C libraries
in {\bf K-OS}.

We plan to finish this part by $5/18$.


\subsection{\textbf{Process Management}}
Once we have some basic memory management, we may create some notion
of process. A data structure is necessary to maintain the Task
State Segment (TSS). With virtual memory, each process will also keep
a copy of the page directory for its own memory space.

We want to support some multitasking. At low level, we will need to
implement Context Switch. This normally involves adding TSS to GDT,
manipulating Task Register (TR) and far jumping to a TSS selector. We
also want some high level mechanisms for process synchronization. 
Simple primitives like semaphores should be sufficient. We need
several queues for processes of different running status, and
functions like $sleep()$ and $wakeup()$ to manipulate these queues.

For scheduling, we will use Round-Robin by default. The user can also
assign priorities to processes. Processes with higher priority will
run with larger quantums. Rescheduling happens either when the task
uses up its quantum (caught by timer interrupt) or the task is
finished (in which case the $reschedule()$ function is called
explicitly). 

We hope to finish this part by $5/25$.


\subsection{\textbf{Application}}
We want to be able to run some applications on {\bf K-OS}. The tricky
part is how to load the process into memory. We might need to
implement some sort of linker/loader, but this should be kept as simple
as possible. 

Application programs will interface with the kernel via system
calls. These are captured by interrupts. Each system call will be
implemented in some higher level language with an assembly
wrapper. 

We will finish this part by $5/30$.


\subsection{\textbf{File System} (Optional)}
We might never get to this part, but it's OK, since there is little
really exciting about a toy file system and it is way too complicated
to implement a real one. 
So we dont want to spend too much time on this part. The hope is that we
can port the DOS FAT12 file system to {\bf K-OS}. We will most likely
keep our file system on a floppy. Since BIOS disk calls are not
available under the protected mode, we need to minimize disk
access. Probably it's a good idea to load the whole floppy image into
memory when booting up.


\section{\textbf{\sc{Summary}}}
With any luck we will have a working version of {\bf K-OS} by $5/30$. 
We might be overly ambitious about this project. Along the way
some features will probably be dropped. And we might find new features
more exciting, or realistic to implement. But this proposal shall serve as
a pretty complete guideline for how we will approach to implement
{\bf K-OS}. The self-imposed deadlines are summarized as following:

\begin{itemize}
\item {\bf 4/25}: Boot Sector.
\item {\bf 5/2}: Interrupt Handler.
\item {\bf 5/11}: Memory Management.
\item {\bf 5/18}: I/O.
\item {\bf 5/25}: Process Management.
\item {\bf 5/30}: Applications.
\item {\bf Optional}: File System.



\end{itemize}



\end{document}
