\documentclass[dvips,11pt]{article}

\usepackage{graphicx}      % extended graphics package
\usepackage{epsfig}        % wrapper for graphicx package
%\usepackage{bbm}
\usepackage{amsmath,amsthm,amsfonts,latexsym}
\newcommand{\etal}{{\it et al}.$\:$}
\newcommand{\eg}{{\it e}.{\it g}.$\:$}
\newcommand{\cf}{{\it cf}.$\:$}
\newcommand{\ie}{{\it i}.{\it e}.$\:$}
\setlength{\oddsidemargin}{0in}
\setlength{\textwidth}{5.8in}
\setlength{\topmargin}{0in}
\setlength{\textheight}{8in}
\setlength{\parskip}{.10in}


\newtheorem{theorem}{Theorem}[section]
\newtheorem{lemma}[theorem]{Lemma}
\newtheorem{proposition}[theorem]{Proposition}
\newtheorem{corollary}[theorem]{Corollary}
\newtheorem{definition}[theorem]{Definition}

%\newenvironment{proof}[1][Proof]{\begin{trivlist}
%\item[\hskip \labelsep {\bfseries #1}]}{\end{trivlist}}

%\newenvironment{definition}[1][Definition]{\begin{trivlist}
%\item[\hskip \labelsep {\bfseries #1}]}{\end{trivlist}}
\newenvironment{example}[1][Example]{\begin{trivlist}
\item[\hskip \labelsep {\bfseries #1}]}{\end{trivlist}}
\newenvironment{remark}[1][Remark]{\begin{trivlist}
\item[\hskip \labelsep {\bfseries #1}]}{\end{trivlist}}

%\newcommand{\qed}{\nobreak \ifvmode \relax \else
%\ifdim\lastskip<1.5em \hskip-\lastskip
%\hskip1.5em plus0em minus0.5em \fi \nobreak
%\vrule height0.75em width0.5em depth0.25em\fi}

\newcommand{\kos}{{\bf K-OS}}


\newcounter{counter}

\begin{document}

\title{\bf{\normalsize{ON THE DESIGN AND IMPLENTATION OF K-OS}}}

\author{\it{\small{Kai Chen}}}
\date{}


\maketitle

\begin{abstract}
{\noindent \kos \ ({\bf K}ai's {\bf O}perating {\bf S}ystem), pronounced
  as {\it chaos}, is designed for the Intel IA-32 Architecture. The
  current version (v0.01) provides fully featured interrupt handling,
  supports terminal I/O and multi-tasking, implements basic virtual
  memory with paging and allows multiple user logins and switching
  between virtual desktops. In the near future it will load user
  applications, implement a few device drivers and a filesystem, and
  provide system calls, API, and an interface to the C runtime
  library. In this paper we describe the design and implementation of
  \kos, and suggest future extensions.
}
\end{abstract}


\section{\textbf{\sc{Introduction}}}
With the original goals \footnotemark[1] in mind, we successfully
designed and implemented a reasonably sophisticated operating system
from scratch. The system provides a rather complete set of features
enjoyed by a real OS, and is designed in such a way that it can be
easily extended and any future work will be more than pleasant.
During the development we studied the source code of various
existing OS' including Linux v0.01 and CHAOS \footnotemark[2], as well
as the Intel Architecture Manuals and numerous online system
resources. We gained much knowledge and experience with hardware
interfacing and low-level system design and development.

In the next section we are pleased to present the design and
implementation details of the current version of \kos (v0.01).



\footnotetext[1]{
Please refer to {\it K-OS PROJECT PROPOSAL}, by {\it Kai Chen}, Spring
2003. 
}
\footnotetext[2]{
CHAOS, by {\it Cheng Hu} and {\it Alex Adriaanse}, (Spring 2002)
was also written for a CS134c project at California Institute of
Technology. The naming was coincidental and the two systems follow
rather different design philosophies, but we did get much insight from
the CHAOS project, and was inspired by its success.
}

\section{\textbf{\sc{Implementation}}}
The current version of \kos \ is composed of a boot sector, the
interrupt handling mechanism, the I/O routines, a memory manager, a
task manager, and miscellaneous features such as virtual desktops and
user authentication. We basically finished all of the projected
milestones in the proposal (which we believed to be a rather ambitious
one), and added a more sophisticated virtual memory manager and various
features. 

In this section we go through each of the components in detail.
To help better understand the design and implementation decisions
and issues, we also provide a self-contained introduction to the
basics of OS development. 


\subsection{\textbf{Boot Sector}}
When the machine is turned on, control is in BIOS. It first performs
the Power-On-Self-Test (POST), and if nothing goes wrong, it selects a
boot device (floppy, CD ROM, harddrive ...) and copies the boot sector
to memory location 0x7C00. The control is then passed to the boot
sector. 

For the Intel Architecture, a boot sector has to be exactly 512 bytes in
size, and must end with 0xAA55.

The first thing \kos \ boot sector does is that it tests the CPU to
make sure we have 80386 or above. This is done by reading bits 12 - 15
in the flags.

Next we switch to the {\it protected mode}. The major differences
between {\it protected mode} and {\it real mode} are the segment
registers and the interrupts. (We will discuss the latter in the next
section.) In {\it real mode}, it is 16-bit addressable, and we
compute the address with:
\begin{verbatim}
      seg:off = seg*16 + off
\end{verbatim}
{\it Protected mode} is 32-bit addressable. Segment registers are {\it
  selectors} that selects entries in the {\it Global Descriptor Table}
  (GDT). 

In a more precise sense, segment registers are really composed of 4
registers: selector, base, limit, and attributes. In {\it real mode},
the register value is stored as the selector, and value*16 is stored
as the base. In {\it protected mode}, the value is used to fetch a
{\it descriptor}, and the {\it descriptor} is unpacked into the 4
registers. 

Once understanding these basics, we can switch to the 
{\it protected mode}. The first thing we need to do is to set up the
GDT. The GDT is a vector of 8-byte descriptors, which contains 32 bits
for base, 20 bits for limit, and 12 bits for descriptor type.

In \kos, the GDT has the following layout:
\begin{verbatim}
      gdtr:
         dw GDT_LIMIT      ; limit = GDT size - 1
         db GDT_BASE       ; base = base address of GDT
      null_sel:            ; null selector, each entry 0
         ...
      code_sel:            ; code selector
         ...
      data_sel:            ; data selector, also for stack segment
         ...
      tss0_sel:            ; after the data selector are the selectors
         ...               ; for task state segments (tss)
      tss1_sel:
         ...
      ...
\end{verbatim}
We made the arbitrary decision to put GDT at memory location
0x500. And statically allocate enough memory for all the
selectors. The code selector is used by the code segment, and the data
selector is used by the data segment and the stack segment. After those,
we have the selectors for Task State Segments (TSS). (More on this
later.) 

To load the GDT register, we simply call:
\begin{verbatim}
      LGDT    [gdtr]
\end{verbatim}

Next we enable the A20 address line. Gate A20 is the 21st address
line. With 20 address lines we can access up to 1Mb of memory. Any
reference beyond that will cause memory wrap around. A20 is ANDed with
some external programmable device (the keyboard controller, to be more
precise, which happens to have an extra pin available). To enable A20,
we communicate with the keyboard controller at ports 0x60, 0x64.

Now we load the kernel into memory. We obtain the kernel size when
compiling, and while still in {\it real mode}, we have access to the
BIOS disk call (INT 13). We only need to set up the registers
properly:
\begin{verbatim}
      AH -- BIOS function. (0x02) 
      AL -- Number of sectors to read
      ES:BX -- segment:offset, memory location
      CH -- Track number
      CL -- Starting sector
      DH -- Head number
      DL -- Drive number
\end{verbatim}

We repeatedly read in sectors and proceed to new tracks or cylinders
when necessary, until we have loaded the whole kernel. In \kos, the
kernel is loaded at memory location 0x10000. 

The actual code to switch to {\it protected mode} is straightforward,
we simply need to set bit 1 of CR0:
\begin{verbatim}
      MOV   EAX, CR0
      OR    AL, 1
      MOV   CR0, EAX
\end{verbatim}

At this point, the
registers still contain old 16-bit segment values. We flush them with
the new GDT selectors, and perform several short jumps to empty the
prefetched instructions. Finally, we are ready to execute the kernel:
\begin{verbatim}
      JMP   CODE_SEL : KERNEL_ADDR
\end{verbatim}



\subsection{\textbf{Interrupts}}
Here we have 3 goals:
\begin{itemize}
\item Set up the {\it Interrupt Descriptor Table} (IDT),
\item Define and install the {\it Interrupt Service Routines} (ISR), and
\item Remap and enable the {\it Programmable Interrupt Controller} (PIC).
\end{itemize}

The IDT is fairly similar to GDT. It's a vector of 8-byte descriptors
associating interrupts with their service routines. We have at most
256 IDT entries since there are at most 256 interrupts on a PC.

Each descriptor mainly contains the address to the corresponding
interrupt handler. It also contains attribute information such as a
present bit, the DPL priviledge level, and a selector. Loading IDT is
very similar to loading GDT:
\begin{verbatim}
      LIDT   [idtr]
\end{verbatim}
where idtr contains the base and limit of the IDT.

We define a separate service routine for each interrupt. The real
handlers are implemented in C, but we provide an assembly wrapper to
conveniently signal EOI and do IRET.

We have 3 highlevel handlers of interest:

\begin{itemize}
\item {\bf Keyboard Interrupt Handler}: Read in the scancode, translate that
  into ASCII, and write the ASCII to the keyboard buffer.
\item {\bf Timer Interrupt Handler}: Decrease the preemption timeout
  counter, if zero, it calls the scheduler. (More on this later.)
\item {\bf Default Interrupt Handler}: This is a panic function, which
  display a Win98-style blue screen with the interrupt number.
\end{itemize}

Most of the low level wrappers will simply store the interrupt number
and call the panic function. The code has the following structure:

\begin{verbatim}
      push all registers
      store interrupt number
      call highlevel handler
      signal EOI
      pop all registers
      IRET
\end{verbatim}

We mentioned earlier that one major difference between {\it real mode}
and {\it protected mode} is in interrupts. 

Our machine has 16 hardware
interrupts, called {\it Interrupt Request} 
(IRQ). These IRQs are divided into two blocks, each associated with a
{\it Programmable Interrupt Controller} (PIC).

In {\it real mode}, IRQs 0 - 7 are mapped to INTs 08h - 0Fh, and IRQs
8 - 15 are mapped to INTs 70h - 77h.

In {\it protected mode}, this interferes with the software
interrupts. So we need to remap PIC.

So we have two PICs. PIC0 is master, and is associated with IRQ0 -
IRQ7. PIC1 is slave, and is associated with IRQ8 - IRQ15.

To program PIC, we communicate with the master and slave controllers
by sending {\it command words}. There are {\it Interrupt Command Word}
(ICW) and {\it Optional Command Word} (OCW). We briefly describe the
usage of each:

\begin{itemize}
\item {\bf ICW1}: specifies whether ICW4 is sent
\item {\bf ICW2}: remapping information
\item {\bf ICW3}: specifies how master and slave are connected
\item {\bf ICW4}: specifies if the CPU expects EOI
\item {\bf OCW1}: masks off unused IRQs
\item {\bf OCW2}: signals EOI
\end{itemize}

Once the PIC is enabled. The CPU starts to accept interrupts.

\subsection{\textbf{I/O}}
\kos \ takes input from the keyboard and outputs to the screen.

When there is a keystroke, a keyboard IRQ is fired, and control is
passed to our keyboard event handler. The handler first reads the
scancode from the keyboard controller's data port. It translates it
into ASCII and writes it to the keyboard buffer. We also handles
breakcode to manipulate multiple keystrokes. In particular, a status
word is maintained to keep track of whether Ctrl, Alt, or Shift is
pressed. We also handle upper and lower cases, and even the LEDs on
the keyboard.

The printing is a bit more interesting. At low level, we write
directly to the video memory. In x86, the video screen is memory
mapped to 0x6B000. The screen is divided into 25 rows and 80
columns. Each character is represented by 2 bytes, one for the ASCII
value, and one for the attributes, which specifies the foreground and
background color.

So printing onto the screen is fairly straightforward. For example, the
following routine outputs a char at a specific location on the screen:

\begin{verbatim}
   void kputc(char c, int row, int col, unsigned char attr)
   {
      unsigned char *vidmem = (unsigned char *) 0x6B000;
      int pos = row*80*2 + col*2;
      
      vidmem[pos] = c;
      vidmem[pos+1] = attr;
   }
\end{verbatim}

Of course we are not satisfied with such primitive printing
routines. We would like to have a kernel version of the printf
function. By Linux convention, let's call it printk:
\begin{verbatim}
      printk(char *fmt, ...)
\end{verbatim}

For a variable argument function like this, we need to figure out the
pointer to the first argument after *fmt. 

Once we have the argument list, the real printing is carried out by
the function:
\begin{verbatim}
      vprintk(char *fmt, void *args)
\end{verbatim}

Right now we support 4 formats: signed decimal (\%d), unsigned
hexadecimal (\%x), string (\%s), char (\%c).

To compute the argument list, Linux uses a set of macros: 
\begin{verbatim}
      va_start, va_end, va_list ... 
\end{verbatim}
\kos \ takes a simplified approach by using the following piece of
assembly code: 
\begin{verbatim}
   printk:
      PUSH       EBP
      MOV        EBP, ESP
      MOV        EAX, EBP       ; EBP points to EBP
      ADD        EAX, 12        ; EBP+4 points to return addr
      PUSH       EAX            ; EBP+8 points to the first arg *fmt
      PUSH dword [EBP+8]        ; EBP+12 points to the second arg
      CALL       vprintk
      MOV        ESP, EBP
      POP        EBP
      RET
\end{verbatim}

Also, at low level, the cursor is nothing more than a flashing spot on
the screen. We would like it to indicate the position the next char is
printed to. Hence we need to manipulate the cursor. We do this by
sending a command to CRT\_CTRL register (0x03D4), and then read the
address from or write the address to CRT\_DATA register
(0x03D5). Finally we translate the address into a position on the
screen:
\begin{verbatim}
      addr = row*80 + col;
\end{verbatim}


\subsection{\textbf{Memory Management}}

\kos \ supports virtual memory with paging. We first review the basics
of paging.

A page directory is a vector of 4-byte page table specifiers, each
taking up to 4Kb to store. A page table is a vector of 4-byte page
specifiers. One page table maps 4Mb of memory and takes up to 4Kb to
store. A page frame is 4Kb of contiguous memory. The starting address
of a page frame must be 4Kb aligned, which means the low 12 bits are
all zero.

A page directory entry or page table entry may contain a pointer to a
physical page. Since page frames are 4Kb aligned, it only takes 20
bits to specify an address. The low 12 bits in a PDE or PTE are thus
used to store attribute information. Most interesting are the present
bit and the permission bits.

The following example shows how to set up paging. It puts the page
directory at 0x70000 and the page tables right after the page
directory. It also identity maps first 4Mb of the
memory. The actual setup in \kos \ is a bit more complicated, which we
will discuss later:
\begin{verbatim}
   void setup_paging()
   {
      Unsigned long * page_dir = (unsigned long *) 0x70000;
      unsigned long * page_table = (unsigned long *) 0x71000;
      unsigned long addr = 0;

      for(i=0; i<1024; i++){
         page_table[i] = addr |  3;
         addr += 4096;
      }
      page_dir[0] = page_table | 3;

      for(i=0; i<1024; i++){
         page_dir[i] = 0 | 2;
      }
   }
\end{verbatim}

In \kos, physical memory is organized in terms of pages. A byte-vector
{\tt page\_use\_count} is used to keep track of the usage of the pages.

We chose to put the page directory at 0x70000, and the page tables
follow the page directory. We also chose to put the kernel heap at
0x100000.

We played a nice trick here to point the last entry of the page
directory back to itself. In this way, the page directory looks like a
page table to the CPU, and a page table looks like a normal
page. Hence we can refer to the page directory and page tables
themselves via virtual memory:
\begin{verbatim}
   PAGE_SELF = 1023
   PAGE_DIR_VADDR = (PAGE_SELF<<22) | (PAGE_SELF<<12)
   PAGE_TABLE_VADDR = (PAGE_SELF << 22)
\end{verbatim}

Next we go through the components of the memory manager.

At the lowest level are the page allocator and deallocator. The page
allocator looks for a free page in the {\tt page\_use\_count}
vector. Once it finds one, it increments the count and returns the
physical address of the page. Deallocation is the natural opposite. 

We also need a memory map function to map physical memory to virtual
address. It basically computes the page table index and insert the
physical address into the page table:
\begin{verbatim}
   mmap(vaddr, paddr, attr)
   {
      pde = vaddr >> 22;
      pte = vaddr >> 12;
      
      check page_dir[pde] present;
      page align paddr (so we dont corrupt attribute bits);
      
      page_table[pte] = paddr;
   }
\end{verbatim}

Our low level allocator is the {\tt morecore} function. It allocates a
page, performs memory map, and returns the virtual address of the
allocated memory. Here memory are allocated in 4Kb blocks. Also, there
is no corresponding deallocator, since we do not allow downsizing the
kernel heap.
\begin{verbatim}
   morecore(int size)
   {
      paddr <- allocate enough pages;
      move along kernel heap to get next available vaddr;
      mmap(vaddr, paddr, attr);
      return vaddr;
   }
\end{verbatim}

Our highlevel allocator is the familiar {\tt kmalloc}. Here memory is
organized in a linked list of blocks. Each block contains a {\it
  header} which contains information about the size of the block,
whether it is being used, and also a pointer to the next block.
\begin{verbatim}
   struct header {
      size_t size;
      struct header *next;
      unsigned magin : 31;
      unsigned used : 1;
   }
\end{verbatim}

Upon request, {\tt kmalloc} first goes through the list of blocks to
check if there is a free block that is large enough. If so, it splits
the block into two to save extra memory, marks the used block and
returns its address. If not found, it calls {\tt morecore} to allocate
a new block first.
\begin{verbatim}
   kmalloc(size_t size)
   {
      look for a free block that is large enough;
      
      if(found){
         split block into two and save extra memory;
         mark and return the block with right size;
      }

      if(not found){
         block = morecore();
         split block into two and save extra memory;
         mark and return the block with right size;
      }
   }
\end{verbatim}

Our deallocator {\tt kfree} does the opposite. It marks the block as
unused and combines the adjacent free blocks to reduce fragmentation:
\begin{verbatim}
   kfree(void *p)
   {
      find the block in the list;
      return error if not found;
      mark block as free;
      combine adjacent free blocks;
   }
\end{verbatim}

In summary, in the allocation, we allocate physical pages,
map them to virtual address, and then reorganize pages of
virtual memory into a linked list of finer blocks. In deallocation, we
mark a block as free and combine the adjacent free blocks.


\subsection{\textbf{Task Management}}

Here we discuss the notion of process, context switches, and
scheduling in \kos.

In x86, the notion of process is captured by the {\it Task State
  Segment} (TSS). A TSS has the following structure:
\begin{verbatim}
   struct tss {
      unsigned short back_link, reserved0;
      unsigned long esp0;
      unsigned short ss0, reserved1;
      unsigned long esp1;
      unsigned short ss1, reserved2;
      unsigned long esp2;
      unsigned short ss2, reserved3;
      unsigned long cr3, eip, eflags;
      unsigned long eax, ecx, edx, ebx, esp, ebp, esi, edi;
      unsigned short es, reserved4;
      unsigned short cs, reserved5;
      unsigned short ss, reserved6;
      unsigned short ds, reserved7;
      unsigned short fs, reserved8;
      unsigned short gs, reserved9;
      unsigned short ldt, reserved10;
      unsigned short tflag, iomap;
   }
\end{verbatim}

Basically entries in TSS reflects the initial state of the registers
of a task. Before any context switch, we at the very least need to set
up the following:
\begin{itemize}
\item CR3: set to the kernel page directory, until we create a
  separate address space for each task.
\item EIP: set to the entry point of the task.
\item Segment Registers: CS set to the code selector; DS, ES, FS, GS,
  SS set to the data selector.
\item ESP: set to the top of the stack we allocated for the task,
  double word aligned.
\end{itemize}

Once the TSS is properly set up, we insert it into the GDT. Each GDT
entry for TSS is just like a normal GDT entry. It contains the base
address of the TSS, its limit, and type and protection informations. A
normal TSS has type 9, and a busy TSS has type 11. The TSS entries
come right after the code descriptor. In the current version, \kos \
statically allocates 16 tasks so we have 16 TSS descriptors in the
GDT. 

After installing the TSS in the GDT, we can perform a context
switch. We simply do a far jump to the TSS selector. For example, TSS0
has selector 0x18, TSS1 is 0x20 ...

In \kos, the task structure contains its TSS, a TSS selector, a status
word, and a task pointer for queue implementations:

\begin{verbatim}
   struct task_t {
      tss_t tss;
      unsigned short tss_sel;
      enum{
         TASK_RUNNING = 0,
         TASK_INTERRUPTIBLE = 1,
         TASK_UNINTERRUPTIBLE = 2,
         TASK_NON_EXIST = 3
      } status;
      struct task_t *next;  // for runqueue, wait_queue �
   }
\end{verbatim}

The current scheduler implements a simple round-robin algorithm. We have a
timeout counter for 
preemption. In the timer interrupt, the counter is decreased and if it
reaches 0, the scheduler is invoked. The scheduler simply performs a
context switch to the next task in the runqueue. 

The current version has some issues in context switches, we will
discuss more about this in a later section.


\subsection{\textbf{Miscellaneous}}

\kos \ implements 4 virtual desktops. Each VD has its own command
buffer, so this creates 4 separate work spaces for the user. The
virtual desktops will become increasingly useful when we have multiple
tasks running.

The shell runs as a separate thread. It constantly reads bytes in from
the keyboard buffer, writes them to the current virtual desktop's
command buffer, and interprets the command.

We also provide basic user authentications. Of course this will only
be useful when we have a real filesystem. In the current version, on start
up, the user should log in as root. The root can then create new users
and delete existing users (except for the root itself). Normal users
do not have such priviledges, but have access to all other features of
the system.


\section{\textbf{\sc{Known Bugs}}}

\kos \ is a fairly stable system. But we are having some issues with
the task manager. At this point, the context switch generates a
general protection fault 
\footnote[3]{
Before, we also got a CPU triple fault in some instances. As this
document was being written, David Moore noticed a typo in the code
which solved that problem.
}.
According to the Intel Architecture Manual Volumn III, a general
protection fault (\#GP) occurs during task switch when:
\begin{itemize}
\item The TSS selector does not reference the GDT or it's not within
  the limit of the GDT.
\item The TSS descriptor is busy.
\end{itemize}
In theory neither of these should have happened. We suspect this is
due to some overlooked technical details, and plan to write a simple
kernel debugger to dump memory, stack and register values. 



\section{\textbf{\sc{Extensions}}}

\kos\ was designed in such a way that it can be easily extended. Each
of the existing features has a pretty complete implementation. For
instance, the keyboard module handles any combination of keystrokes
people will ever want to use, although many of them are not being used
in the current version. The standard functions such as 
{\tt printk} and {\tt kmalloc} greatly ease the rest of the
development. And we also have a decent support for the string library.

In the near future we plan to provide an interface to the C runtime
library. In particular, we will implement the \kos\ version of 
{\tt printf}, {\tt scanf}, {\tt malloc}, {\tt free}, and the string
operations. 

We also want to support several system calls. Since we have a
full support for the interrupts, and already have most of the utility
functions, this part should be relatively straightforward.

We already have an implementation for the DOS FAT12 file system, which
can be ported to \kos\ fairly easily. However, since we do not have
access to the 
BIOS disk functions under {\it protected mode}, we need to implement
our own disk driver first. The other option is to use direct hardware
access, but that is painful and does no good in the long run.

Given the support of system calls, the C runtime library, and a file
system, we can start to write and load applications for \kos. From
there it's not far away from being a real world operating system.

\section{\textbf{\sc{Conclusion}}}

The \kos\ project has been a great success. In 8 we weeks designed and
implemented an operating system with most of the basic features a
modern OS enjoys. And the design allows a fairly sustainable
development. During the course of the project, we obtained tons of
knowledge in system design, and gained much experience with low level
development. It has been a lot of fun working on this project. This
report is not a final one, and there will never be a final one, since
the work will continue.


\begin{center}
{\bf Acknowledgements}
\end{center}

I would like to thank Professor Jason Hickey for teaching me
this material, and would like to thank David Moore, Justin Smith, and
Christian Tapus for their help and instructions on this project. I'm
also grateful to everyone who have used \kos,
provided comments and suggestions, or showed interest in the project.


\end{document}







